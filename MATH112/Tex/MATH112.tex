\documentclass[a4paper]{book}
\usepackage{amssymb}
\usepackage{amsmath}
\def\arraystretch{2.4}
\author{Yahia Gaber}
\title{MATH112}

\begin{document}

\maketitle
\tableofcontents

\chapter{Integration}

Integration is the anti-derivative of a function.

\[\frac{d}{dx}(F(x)) = f(x) \leftrightarrow \int f(x) = F(x)\]

\section{Basic integration formulas}

\begin{center}

  \begin{tabular}{|c|c|}

    \hline
    $\displaystyle{\int}$ & $\displaystyle{\frac{d}{dx}}$ \\ \hline \hline
    $\displaystyle{x^n \cdot dx}$ & $\displaystyle{\frac{x^{n + 1}}{n + 1}}$ \\ \hline
    $\displaystyle{\frac{1}{f(x)} \cdot dx}$ & $\displaystyle{\ln(f'(x))}$ \\ \hline
    $\displaystyle{a^x \cdot dx}$ & $\displaystyle{\frac{a^x}{\ln(a)}}$ \\
    \hline

  \end{tabular}

\end{center}

\section{Basic trignometric identites}

\begin{center}

  \begin{tabular}{|c|c|}

    \hline
    $\displaystyle{\sin^2(x) + \cos^2(x)}$ & $\displaystyle{1}$ \\ \hline \hline
    $\displaystyle{\cosh^2(x) - \sinh^2(x)}$ & $\displaystyle{1}$ \\ \hline \hline
    $\displaystyle{\sec^2(x) - \tan^2(x)}$ & $\displaystyle{1}$ \\ \hline \hline
    $\displaystyle{\tanh^2(x) + \operatorname{sech}^2(x)}$ & $\displaystyle{1}$ \\ \hline \hline
    $\displaystyle{\csc^2(x) - \cot^2(x)}$ & $\displaystyle{1}$ \\ \hline \hline
    \hline

  \end{tabular}

\end{center}

\end{document}
