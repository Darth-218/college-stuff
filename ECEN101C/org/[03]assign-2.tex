% Created 2024-03-07 Thu 16:38
% Intended LaTeX compiler: pdflatex
\documentclass[11pt]{article}
\usepackage[utf8]{inputenc}
\usepackage[T1]{fontenc}
\usepackage{graphicx}
\usepackage{longtable}
\usepackage{wrapfig}
\usepackage{rotating}
\usepackage[normalem]{ulem}
\usepackage{amsmath}
\usepackage{amssymb}
\usepackage{capt-of}
\usepackage{hyperref}
\author{Yahia Gaber}
\date{\today}
\title{}
\hypersetup{
 pdfauthor={Yahia Gaber},
 pdftitle={},
 pdfkeywords={},
 pdfsubject={},
 pdfcreator={Emacs 29.2 (Org mode 9.7)}, 
 pdflang={English}}
\begin{document}

\tableofcontents

Student Name: Yahia Hany Gaber |
Student ID: 231000412
\section{Assignment \#2}
\label{sec:org63269cc}

\subsection{Question 1}
\label{sec:org3e6dd82}

Find \(v_1\) and \(v_g\) when \(v_o = 5V\).
\subsection{Question 2}
\label{sec:orgbc8c09a}

\subsubsection{(a)}
\label{sec:org4c67fcd}

\begin{itemize}
\item 26 and 10 \(\Omega\) are in series \(\to 26 + 10 = 36 \Omega\)

\item 36 and 18 \(\Omega\) are in parallel \(\to \frac{36 \times 18}{36 + 18} = 12 \Omega\)

\item 12 and 6 \(\Omega\) are in series \(\to 12 + 6 = 18 \Omega\)

\item 18 and 36 \(\Omega\) are in parallel \(\to \frac{36 \times 18}{36 + 18} = 12 \Omega\)
\end{itemize}

$$\therefore \boxed{R_{ab} = 12 \Omega}$$
\subsubsection{(b)}
\label{sec:org2cbb0b4}

\begin{itemize}
\item 12 and 18 \(\Omega\) are in series \(\to 12 + 18 = 30 \Omega\)

\item 30 and 10 \(\Omega\) are in parallel \(\to \frac{30 \times 10}{30 + 10} = 7.5 \Omega\)

\item 7.5 and 15 \(\Omega\) are in parallel \(\to \frac{7.5 \times 15}{7.5 + 15} = 5 \Omega\)

\item 5 and 20 \(\Omega\) are in parallel \(\to \frac{5 \times 20}{5 + 20} = 4 \Omega\)

\item 4 and 16 \(\Omega\) are in series \(\to 4 + 16 = 20 \Omega\)

\item 20 and 30 \(\Omega\) are in parallel \(\to \frac{20 \times 30}{20 + 30} = 12 \Omega\)

\item 4, 12 and 14 \(\Omega\) are in series \(\to 4 + 12 + 14 = 30 \Omega\)
\end{itemize}

$$\therefore \boxed{R_{ab} = 30 \Omega}$$
\subsubsection{(c)}
\label{sec:orgc871c4c}

\begin{itemize}
\item 500 and 1500 \(\Omega\) are in parallel \(\to \frac{500 \times 1500}{500 + 1500} = 375 \Omega\)

\item 375 and 750 \(\Omega\) are in parallel \(\to \frac{375 \times 750}{375 + 750} = 250 \Omega\)

\item 250 and 250 \(\Omega\) are in series \(\to 250 + 250 = 500 \Omega\)

\item 500 and 2000 \(\Omega\) are in parallel \(\to \frac{500 \times 2000}{500 + 2000} = 400 \Omega\)

\item 400 and 1000 \(\Omega\) are in series \(\to 400 + 1000 = 1400 \Omega\)
\end{itemize}

$$\therefore \boxed{R_{_{ab}} = 1400 \Omega}$$
\subsubsection{(d)}
\label{sec:orge49f39e}

\begin{itemize}
\item 60 \(\Omega\) is short-circuited and ignored

\item 30 and 18 \(\Omega\) are in series \(\to 30 + 18 = 48 \Omega\)

\item 48 and 16 \(\Omega\) are in parallel \(\to \frac{48 \times 16}{48 + 16} = 12 \Omega\)

\item 12 and 28 \(\Omega\) are in series \(\to 12 + 28 = 40 \Omega\)

\item 40 and 40 \(\Omega\) are in parallel \(\to \frac{40 \times 40}{40 + 40} = 20 \Omega\)

\item 20 and 20 \(\Omega\) are in series \(\to 20 + 20 = 40 \Omega\)

\item 40 and 24 \(\Omega\) are in parallel \(\to \frac{40 + 24}{40 + 24} = 15 \Omega\)

\item 25, 15 and 10 \(\Omega\) are in series \(\to 25 + 15 + 10 = 50 \Omega\)

\item 50 and 50 \(\Omega\) are in parallel \(\to \frac{50 \times 50}{50 + 50} = 25 \Omega\)
\end{itemize}

$$\therefore \boxed{R_{ab} = 25 \Omega}$$
\subsection{Question 3}
\label{sec:org6239fc3}

The equivelent resistance for the circuit is:

\begin{itemize}
\item 1 and 13 \(\Omega\) are in series \(\to 15 + 12 + 13 = 40 \Omega\)

\item 520 \(\Omega\) are in parallel \(\to \frac{5 \times 20}{5 + 20} = 4 \Omega\)

\item 64 \(\Omega\) are in series \(\to 6 + 4 = 10 \Omega\)

\item 1 40 \(\Omega\) are in parallel \(\to \frac{10 \times 40}{10 + 40} = 8 \Omega\)

\item 82 \(\Omega\) are in series \(\to 8 + 2 = 10 \Omega\)
\end{itemize}

$$\therefore \boxed{R_{eq} = 10 \Omega}$$

The value of \(i_g\):

$$I = \frac{V}{R} \to i_{g} = \frac{V}{R_{eq}} = \frac{125}{10} = 12.5A$$

Using current division law to find \(I_{6 \Omega}\):

$$I_{6\Omega} = i_{g} \times \frac{40}{50} = 10A$$

Dividing that current (\(I_{6\Omega}\)) to find \(i_o\):

$$i_{o} = I_{6\Omega} \times \frac{5}{25} = 2A$$

$$\therefore \boxed{i_o = 2A}$$
\subsection{Question 4}
\label{sec:orgc24084c}

\begin{enumerate}
\item \ldots{}

Using current division to find \(I_{6\Omega}\):

$$I_{6\Omega} = 2.4 \times \frac{30}{15} = 4.8A$$

Dividing the current \(I_{6\Omega}\) to find \(i_o\):

$$i_o = I_{6\Omega} \times \frac{10}{100} = 0.48A$$

$$\boxed{i_{o} = 0.48}$$

Using current division to find \(I_{20\Omega}\):

$$I_{20 \Omega} = 2.4 \times \frac{15}{30} = 1.2A$$

Using Ohm's law to find \(v_o\):

$$V = I \times R \to v_{o} = I_{20\Omega} \times 20 = 24V$$

\item \ldots{}

The power used by the 6 \(\Omega\) resistor:

$$P = I^2 \times R = (4.8)^2 \times 6 = 138.24 W$$

\item \ldots{}

The power supplied by the current source using KVL on the leftmost mesh:

$$-V_{2.4A} + 20 \times 1.2 + 10 \times 1.2 = 0 \to V_{2.4A} = 36V$$

$$\therefore P_{2.4A} = V \times I = 36 \times 2.4 \to \boxed{P = 86.4 W}$$
\end{enumerate}
\end{document}
