% Created 2024-03-04 Mon 16:33
% Intended LaTeX compiler: pdflatex
\documentclass[11pt]{article}
\usepackage[utf8]{inputenc}
\usepackage[T1]{fontenc}
\usepackage{graphicx}
\usepackage{longtable}
\usepackage{wrapfig}
\usepackage{rotating}
\usepackage[normalem]{ulem}
\usepackage{amsmath}
\usepackage{amssymb}
\usepackage{capt-of}
\usepackage{hyperref}
\author{Yahia Gaber}
\date{\today}
\title{}
\hypersetup{
 pdfauthor={Yahia Gaber},
 pdftitle={},
 pdfkeywords={},
 pdfsubject={},
 pdfcreator={Emacs 29.1 (Org mode 9.7)}, 
 pdflang={English}}
\begin{document}

\tableofcontents

Student Name: Yahia Hany Gaber |
Student ID: 231000412
\section{Assignment \#2}
\label{sec:org108fe0e}

\subsection{Question 1}
\label{sec:orgb791164}

Find \(v_1\) and \(v_g\) when \(v_o = 5V\).
\subsection{Question 2}
\label{sec:orgbe6aaae}

\subsubsection{(a)}
\label{sec:orgf2ee893}

\begin{itemize}
\item 26 and 10 \(\Omega\) are in series \(\to 26 + 10 = 36 \Omega\)

\item 36 and 18 \(\Omega\) are in parallel \(\to \frac{36 \times 18}{36 + 18} = 12 \Omega\)

\item 12 and 6 \(\Omega\) are in series \(\to 12 + 6 = 18 \Omega\)

\item 18 and 36 \(\Omega\) are in parallel \(\to \frac{36 \times 18}{36 + 18} = 12 \Omega\)

\(\therefore R_{ab} = 12 \Omega\)
\end{itemize}
\subsubsection{(b)}
\label{sec:org65c003b}

\begin{itemize}
\item 12 and 18 \(\Omega\) are in series \(\to 12 + 18 = 30 \Omega\)

\item 30 and 10 \(\Omega\) are in parallel \(\to \frac{30 \times 10}{30 + 10} = 7.5 \Omega\)

\item 7.5 and 15 \(\Omega\) are in parallel \(\to \frac{7.5 \times 15}{7.5 + 15} = 5 \Omega\)

\item 5 and 20 Ω are in parallel → \$\frac{5 * 20}{5 + 20} = 4Ω\$Ω
\end{itemize}
\subsection{Question 3}
\label{sec:org5e5eaa4}

\subsection{Question 4}
\label{sec:org27f0d2e}
\end{document}
