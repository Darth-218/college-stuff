\documentclass[a4paper]{book}
\usepackage{relsize}
\usepackage{amssymb}
\usepackage{amsmath}
\usepackage{float}
\usepackage{circuitikz}
\ctikzset{bipoles/length=0.9cm}
\def\arraystretch{1.5}
\author{Yahia Gaber}
\title{ECEN101C}
\begin{document}

\maketitle
\tableofcontents

\chapter{Introduction to electric circuits}

\section{Volt classification}

Electric volt can be classified into three categories: low, Medium and high.

\begin{center}

  \begin{tabular}{| c | c | c |}

    \hline
    Low voltage & Medium voltage & High voltage \\
    \hline
    :1000V & 1kV:50kV & 50kV:500kv \\
    \hline

  \end{tabular}

\end{center}

\section{Electric circuit components}

For an electric circuit to be one some elements need to be present.

\begin{enumerate}

  \item Power supplies:\\
    A power supply is needed to create a potential difference to provide the circuit with a current.

  \item A load:\\
    "loads" are energies converted in the circuit by elements that consume the power generated.

  \item Wires:\\
    Wires are used to connect all the elements of the circuit together (Most commonly made of copper or aluminum).

  \item Switches:\\
    Switches are used to "open" or "close" the circuit in order for current to pass.

\end{enumerate}

\newpage
\section{Basic circuit quantities}

\begin{enumerate}

  \item The electric charge $q$ (C: Coulomb)

  \item The electric current $I$ (A: Ampere):\\
    The current is the rate of change of the quantity of charge in a certain time period.

    \[I = \frac{dq}{dt}\]
    \[\therefore q = \int i(t).dt\]

    Example:

    \[q = 12e^{-12t} \to I = (-12)12e^{-12t}\]

  \item The potential difference (V $=$ J/C: Volt/Joule per Coloumb):\\
    The potential difference is the energy affecting the charge moving it a certain distance.

    \[V = \frac{dw}{dq}\]

  \item The electric energy (Joule)

  \item The electric power (W $=$ J/s: Watt/Joule per second):\\
    Power is the energy consumed/delivered in a certain period of time.

    \[P = \frac{dW}{dt} = I \times V = I^{2} \times R = \frac{V^2}{R}\]

\end{enumerate}

\section{Circuit types}

\begin{enumerate}

  \item DC circuits:\\
    Direct current circuits are circuits with constant voltage and current.

    \begin{center}
      \begin{circuitikz} \draw

        (0, 0) to[battery, o-o] (3, 0)
        ;
      \end{circuitikz}
    \end{center}

  \item AC circuits:\\
    Alternating current circuits are circuits with alternating voltage and current.

    \begin{center}
      \begin{circuitikz} \draw

        (0, 0) to[vsourcesin, o-o] (3, 0)
        ;
      \end{circuitikz}
    \end{center}

\end{enumerate}

\section{Basic circuit elements}

\begin{itemize}

    % TODO Insert example

  \item Passive elements:

    Elements that absorb the power generated by the active elements.

    Some of the common passive elements seen in circuits:

    \begin{center}
      \begin{tabular}{| c | c | c |}

        \hline
        Element & Unit & Found in \\ \hline \hline
        Resistors & Ohm ($\Omega$) & AC and DC \\ \hline
        Electrical coils & Henry (L) & AC only \\ \hline
        Capacitors & Farad (C) & AC only\\
        \hline

      \end{tabular}
    \end{center}

  \item Active elements: \\ Elements that generate power for the circuit.

    \begin{itemize}

      \item Current sources

        \begin{enumerate}

          \item Independant Current sources:

            Independant sources provide constant current intensity.

            \begin{center}
              \begin{circuitikz} \draw

                (0, 0) to[american current source, o-o] (3, 0);

              \end{circuitikz}
            \end{center}

          \item Dependant current sources:

            Dependant sources have variable current intensities and are either:

            \begin{center}
              \begin{circuitikz} \draw

                (0, 0) to[american controlled current source, o-o] (3, 0);

              \end{circuitikz}
            \end{center}

            \begin{itemize}

              \item Voltage controlled ($V_x$)

              \item Current controlled ($I_y$)

            \end{itemize}

        \end{enumerate}

    \end{itemize}

\end{itemize}

\chapter{Basic circuit laws}

\section{Ohm's law}

\begin{center}
  \begin{tabular}{| c |}

    \hline
    $\displaystyle{\mathlarger{\mathlarger{V = I \times R}}}$ \\
    \hline

  \end{tabular}
\end{center}

\subsubsection{Power types and the conventional sign rule}

\begin{itemize}

  \item[-] Power absorbed

    Power is absorbed when the current's direction is into the positive terminal

    \begin{center}
      \begin{circuitikz} \draw

        (3, 0) to[american voltage source, o-o] (0, 0)
        (3, 0) to[short, i=i] (2.3, 0)
        ;
      \end{circuitikz}
    \end{center}

  \item[-] Power supplied

    Power is supplied when the current's direction is into the negative terminal

    \begin{center}
      \begin{circuitikz} \draw

        (3, 0) to[american voltage source, o-o] (0, 0)
        (2.3, 0) to[short, i=i] (3, 0)
        ;
      \end{circuitikz}
    \end{center}

\end{itemize}

\noindent For any balanced (ideal) circuit, the sum of the power consumed equals the sum of power absorbed.

\[\sum P_{abs} = \sum P_{con}\]

\noindent Note: The conventional sign rule is only applied when both the current 
and the voltage are positive and allows the switching of either the direction of 
a current of the terminals of a voltage source in case the magnitude is a negative 
value.

\newpage
\section{Kirchhoff's laws}

\subsection{Kirchhoff's voltage law}

In any loop in a circuit, the sum of voltages across the loop equals zero

\[\sum V_{loop} = 0\]

\subsection{Series connection}

Circuit elements are in series only if the same current intensity passes through them as the voltage is divided between the elements (not equally).\\
Therefore the equivalent resistance of a number of resistors in series is:

\[R_{eq} = \sum_{n = 1}^r R_{n}\]

\subsubsection{Voltage division}

As the voltage is divided between the elements in series in non-uniform quantities, the voltage of each element can be found as the voltage 
is directly proportional with the value of the resistance of each element. Therefore:

\[V_a = V_t \times \frac{R_a}{R_{eq}}\]

\section{Kirchhoff's current law}

\begin{center}

  \begin{tabular}{| l |}

    \hline

      - Junctions: points of connection that connect only two circuit elements.\\

      - Nodes: points of connection that connect more than two circuit elements.\\

    \hline

  \end{tabular}

\end{center}

\noindent At any node, the sum of currents with a direction into the node equals to the sum of
currents with a direction outside the node.

\[\sum I_{in} = \sum I_{out}\]

\subsection{Parallel connection}

Circuit elements are in parallel only if they share the same starting and ending node as the current is 
divided between each element while the voltage remains the same.\\
Therefore the equivalent resistance of a number of resistors in series is:

\[\frac{1}{R_{eq}} = \sum_{n=1}^r \frac{1}{R_n}\]

\subsubsection{Current division}

As the current is devided between the elements in parallel non-uniform quantities, the current through
each element can be found as the current intensity is inversly proportional with the value of the resistance of each element. Therefore:

\[I_a = I_{t} \times \frac{R_b}{R_a + R_b}\]

\subsection{Conductance}

Conductance ($G$) is the receprocal quantity to the electrical resistance and is measured in siemens ($S$)

\[G = \frac{I}{V} = \frac{1}{R}\]

\newpage
\section{Examples}

% TODO Add example from lec

\begin{enumerate}

  \item Find the value of the resistors then verify the power balance.

    \begin{center}
      \begin{circuitikz} \draw

        (0, -2) node[label={left:a}] {}
        (4, -2) node[label={right:c}] {}
        (2, -2) node[label={above:b}] {}
        (0, 0) to[american current source, o-o, l=2A, i=$I_b$] (4, 0)
        (4, 0) to[short] (4, -2)
        (4, -2) to[resistor, -o, l_=$R_b$, a^=-3V+] (2, -2)
        (2, -2) to[resistor, o-, l_=$R_a$, a^=+4V-] (0, -2)
        (0, -4) to[american current source, o-o, l=3A] (0, -2)
        (2, -4) to[short, o-o, i=$I_a$] (0, -4)
        (2, -4) to[resistor, l=$R_c$, a=+12V-] (2, -2)
        (4, -4) to[short, o-o, i=$I_c$] (2, -4)
        (4, -2) to[american current source, o-o, l=1.5A] (4, -4)
        (0, 0) to[short, o-o] (0, -2)
        ;

      \end{circuitikz}
    \end{center}

    KCL at node a:
    \[\sum I_{in} = \sum I_{out} \to I_a + 2 = 3 \to I_a = 1A\]
    \[\therefore R_a = \frac{4}{I_a}\]

    KCL at node b:
    \[\sum I_{in} = \sum I_{out} \to I_b + 1.5 = 2 \to I_b = 0.5A\]
    \[\therefore R_b = \frac{3}{I_b}\]

    KCL at node c:
    \[\sum I_{in} = \sum I_{out} \to I_c + 1.5 = 3 \to I_c = 1.5A\]
    \[\therefore R_c = \frac{12}{I_c}\]

    By solving the three equations: $\to R_a = 4 \Omega, \quad R_b = 6 \Omega, \quad R_c = 8 \Omega$

    Verifying the power balance:
    \[\sum P_{supplied} = \sum P_{abosrbed}\]

    KVL at loop 1:
    \[\Sigma V_{loop} = 0 \rightarrow -V_1 + 4 + 12 = 0 \rightarrow V_1 = 16V\]
    \[\therefore P_{3A} = 16 \times 3 = 48W (supplied)\]

    KVL at loop 2:
    \[\Sigma V_{loop} = 0 \rightarrow -V_2 + 3 - 4 = 0 \rightarrow V_2 = -1V\]
    \[\therefore P_{2A} = 2 \times 1 = 2W (absorbed)\]

    KVL at loop 3:
    \[\Sigma V_{loop} = 0 \rightarrow -V_3 - 12 - 3 = 0 \rightarrow V_2 = -15V\]
    \[\therefore P_{1.5A} = 15 \times 1.5 = 22.5W (absorbed)\]

    Power at $R_a$.
    \[P_{R_a} = 4 \times I_a = 4W (absorbed)\]

    Power at $R_b$.
    \[P_{R_b} = 3 \times I_b = 1.5W (absorbed)\]

    Power at $R_c$.
    \[P_{R_a} = 12 \times I_c = 18W (absorbed)\]

    \[\therefore \Sigma P = -48 + 2 + 22.5 + 4 + 1.5 + 18 = 0\]
    \[\therefore \text{The power balance verified.}\]

    \newpage

  \item Find the equivalent resistance for the circuit given. If a power supply of 100V
    is connected to the circuit across a-b, find the total power absorbed by the circuit

    \begin{center}
      \begin{circuitikz} \draw

        (-0.5, 0) node[label={left:a}] {}
        (-0.5, -3) node[label={left:b}] {}
        (-0.5, 0) to[short, o-] (0, 0)
        (-0.5, -3) to[short, o-] (0, -3)
        (0, 0) to[resistor, l=$25\Omega$, o-o] (2, 0)
        (2, 0) to[resistor, l=$28\Omega$, o-o] (4.5, 0)
        (4.5, 0) to[resistor, l=$30\Omega$, o-o] (7, 0)

        (0, 0) to[resistor, l=$50\Omega$, o-o] (0, -3)
        (2, 0) to[resistor, l=$40\Omega$, o-o] (4.5, -3)
        (2, 0) to[resistor, l=$24\Omega$, o-o] (2, -3)
        (4.5, 0) to[resistor, l=$16\Omega$, o-o] (4.5, -3)
        (6.5, 0) to[short, o-o] (6.5, -3)
        (7, 0) to[resistor, l=$16\Omega$, o-o] (7, -3)

        (0, -3) to[resistor, l=$10\Omega$, o-o] (2, -3)
        (2, -3) to[resistor, l=$20\Omega$, o-o] (4.5, -3)
        (4.5, -3) to[resistor, l=$18\Omega$, o-o] (7, -3)
        ;

      \end{circuitikz}
    \end{center}

  16 $\Omega$ is short circuited $\to \displaystyle{\frac{0 \times 16}{0 + 16} = 0 \Omega}$

    \begin{center}

      \begin{circuitikz} \draw

        (-0.5, 0) node[label={left:a}] {}
        (-0.5, -3) node[label={left:b}] {}
        (-0.5, 0) to[short, o-] (0, 0)
        (-0.5, -3) to[short, o-] (0, -3)
        (0, 0) to[resistor, l=$25\Omega$, o-o] (2, 0)
        (2, 0) to[resistor, l=$28\Omega$, o-o] (4.5, 0)
        (4.5, 0) to[resistor, l=$30\Omega$, o-o] (6.5, 0)

        (0, 0) to[resistor, l=$50\Omega$, o-o] (0, -3)
        (2, 0) to[resistor, l=$40\Omega$, o-o] (4.5, -3)
        (2, 0) to[resistor, l=$24\Omega$, o-o] (2, -3)
        (4.5, 0) to[resistor, l=$16\Omega$, o-o] (4.5, -3)
        (6.5, 0) to[short, o-o] (6.5, -3)

        (0, -3) to[resistor, l=$10\Omega$, o-o] (2, -3)
        (2, -3) to[resistor, l=$20\Omega$, o-o] (4.5, -3)
        (4.5, -3) to[resistor, l=$18\Omega$, o-o] (6.5, -3)
        ;

      \end{circuitikz}
    \end{center}

  30 and 18 $\Omega$ are in series $\to \displaystyle{30 + 18 = 48 \Omega}$

    \begin{center}

      \begin{circuitikz} \draw

        (-0.5, 0) node[label={left:a}] {}
        (-0.5, -3) node[label={left:b}] {}
        (-0.5, 0) to[short, o-] (0, 0)
        (-0.5, -3) to[short, o-] (0, -3)
        (0, 0) to[resistor, l=$25\Omega$, o-o] (2, 0)
        (2, 0) to[resistor, l=$28\Omega$, o-o] (4.5, 0)
        (4.5, 0) to[short, o-o] (6, 0)

        (0, 0) to[resistor, l=$50\Omega$, o-o] (0, -3)
        (2, 0) to[resistor, l=$40\Omega$, o-o] (4.5, -3)
        (2, 0) to[resistor, l=$24\Omega$, o-o] (2, -3)
        (4.5, 0) to[resistor, l=$16\Omega$, o-o] (4.5, -3)
        (6, 0) to[resistor, o-o, l=$48\Omega$] (6, -3)

        (0, -3) to[resistor, l=$10\Omega$, o-o] (2, -3)
        (2, -3) to[resistor, l=$20\Omega$, o-o] (4.5, -3)
        (4.5, -3) to[short, o-o] (6, -3)
        ;

      \end{circuitikz}
    \end{center}

  48 and 16 $\Omega$ are in parallel $\to \displaystyle{\frac{48 \times 16}{48 + 16} = 12 \Omega}$

    \begin{center}

      \begin{circuitikz} \draw

        (-0.5, 0) node[label={left:a}] {}
        (-0.5, -3) node[label={left:b}] {}
        (-0.5, 0) to[short, o-] (0, 0)
        (-0.5, -3) to[short, o-] (0, -3)
        (0, 0) to[resistor, l=$25\Omega$, o-o] (2, 0)
        (2, 0) to[resistor, l=$28\Omega$, o-o] (4.5, 0)

        (0, 0) to[resistor, l=$50\Omega$, o-o] (0, -3)
        (2, 0) to[resistor, l=$40\Omega$, o-o] (4.5, -3)
        (2, 0) to[resistor, l=$24\Omega$, o-o] (2, -3)
        (4.5, 0) to[resistor, l=$12\Omega$, o-o] (4.5, -3)

        (0, -3) to[resistor, l=$10\Omega$, o-o] (2, -3)
        (2, -3) to[resistor, l=$20\Omega$, o-o] (4.5, -3)
        ;

      \end{circuitikz}
    \end{center}

    12 and 28 $\Omega$ are in series $\to \displaystyle{12 + 28 = 40 \Omega}$

    \begin{center}

      \begin{circuitikz} \draw

        (-0.5, 0) node[label={left:a}] {}
        (-0.5, -3) node[label={left:b}] {}
        (-0.5, 0) to[short, o-] (0, 0)
        (-0.5, -3) to[short, o-] (0, -3)
        (0, 0) to[resistor, l=$25\Omega$, o-o] (2, 0)
        (2, 0) to[resistor, l=$40\Omega$, o-o] (4.5, 0)

        (0, 0) to[resistor, l=$50\Omega$, o-o] (0, -3)
        (2, 0) to[resistor, l=$40\Omega$, o-o] (4.5, -3)
        (2, 0) to[resistor, l=$24\Omega$, o-o] (2, -3)
        (4.5, 0) to[short, o-o] (4.5, -3)

        (0, -3) to[resistor, l=$10\Omega$, o-o] (2, -3)
        (2, -3) to[resistor, l=$20\Omega$, o-o] (4.5, -3)
        ;

      \end{circuitikz}
    \end{center}


  40 and 40 $\Omega$ are in parallel $\to \displaystyle{\frac{40 * 40}{40 + 40} = 20 \Omega}$

    \begin{center}

      \begin{circuitikz} \draw

        (-0.5, 0) node[label={left:a}] {}
        (-0.5, -3) node[label={left:b}] {}
        (-0.5, 0) to[short, o-] (0, 0)
        (-0.5, -3) to[short, o-] (0, -3)
        (0, 0) to[resistor, l=$25\Omega$, o-o] (2, 0)
        (2, 0) to[resistor, l=$20\Omega$, o-o] (4.5, 0)

        (0, 0) to[resistor, l=$50\Omega$, o-o] (0, -3)
        (2, 0) to[resistor, l=$24\Omega$, o-o] (2, -3)
        (4.5, 0) to[short, o-o] (4.5, -3)

        (0, -3) to[resistor, l=$10\Omega$, o-o] (2, -3)
        (2, -3) to[resistor, l=$20\Omega$, o-o] (4.5, -3)
        ;

      \end{circuitikz}
    \end{center}


20 and 20 $\Omega$ are in series $\to \displaystyle{20 + 20 = 40 \Omega}$

    \begin{center}

      \begin{circuitikz} \draw

        (-0.5, 0) node[label={left:a}] {}
        (-0.5, -3) node[label={left:b}] {}
        (-0.5, 0) to[short, o-] (0, 0)
        (-0.5, -3) to[short, o-] (0, -3)
        (0, 0) to[resistor, l=$25\Omega$, o-o] (2, 0)
        (2, 0) to[short, o-o] (4, 0)

        (0, 0) to[resistor, l=$50\Omega$, o-o] (0, -3)
        (2, 0) to[resistor, l=$24\Omega$, o-o] (2, -3)
        (4, 0) to[resistor, l=$40\Omega$, o-o] (4, -3)

        (0, -3) to[resistor, l=$10\Omega$, o-o] (2, -3)
        (2, -3) to[short, o-o] (4, -3)
        ;

      \end{circuitikz}
    \end{center}

40 and 24 $\Omega$ are in parallel $\to \displaystyle{\frac{40 \times 24}{40 + 24} = 15 \Omega}$

    \begin{center}

      \begin{circuitikz} \draw

        (-0.5, 0) node[label={left:a}] {}
        (-0.5, -3) node[label={left:b}] {}
        (-0.5, 0) to[short, o-] (0, 0)
        (-0.5, -3) to[short, o-] (0, -3)
        (0, 0) to[resistor, l=$25\Omega$, o-o] (2, 0)

        (0, 0) to[resistor, l=$50\Omega$, o-o] (0, -3)
        (2, 0) to[resistor, l=$15\Omega$, o-o] (2, -3)

        (0, -3) to[resistor, l=$10\Omega$, o-o] (2, -3)
        ;

      \end{circuitikz}
    \end{center}

  15, 25 and 10 $\Omega$ are in series $\to \displaystyle{15 + 25 + 10 = 50 \Omega}$

    \begin{center}

      \begin{circuitikz} \draw

        (-0.5, 0) node[label={left:a}] {}
        (-0.5, -3) node[label={left:b}] {}
        (-0.5, 0) to[short, o-] (0, 0)
        (-0.5, -3) to[short, o-] (0, -3)
        (0, 0) to[short, o-o] (2, 0)

        (0, 0) to[resistor, l=$50\Omega$, o-o] (0, -3)
        (2, 0) to[resistor, l=$50\Omega$, o-o] (2, -3)

        (0, -3) to[short, o-o] (2, -3)
        ;

      \end{circuitikz}
    \end{center}

  50 and 50 $\Omega$ are in parallel $\to \displaystyle{\frac{50 \times 50}{50 + 50} = 25 \Omega}$

    \begin{center}

      \begin{circuitikz} \draw

        (-0.5, 0) node[label={left:a}] {}
        (-0.5, -3) node[label={left:b}] {}
        (-0.5, 0) to[short, o-] (0, 0)
        (-0.5, -3) to[short, o-] (0, -3)

        (0, 0) to[resistor, l=$25\Omega$, o-o] (0, -3)
        ;

      \end{circuitikz}
    \end{center}

  Therefore the equivelent resistance of the circuit:
  \[R_{ab} = 25 \Omega\]

  Therefore the total power absorbed by the circuit if 100V is connected to the circuit is:
  \[P_{ab} = \frac{V^2}{R} = \frac{100^2}{25} = 400W\]
\end{enumerate}

\chapter{Techincal methods for solving electrical circuits}

\section{Mesh analysis}

To find the current in a circuit using mesh analysis:

\begin{enumerate}

  \item Find the number of meshes in the circuit.
  \item Assume a the current's direction in each mesh.
  \item Apply KVL (mesh equations) accross each mesh's elements to find the value of each assumed current.

\end{enumerate}

\subsection{Mesh equations}

\begin{itemize}

  \item[-] The left hand side of the equation:

    The left hand side contains the value of the voltage supplied by a source in a certain mesh.\\
    If the voltage source supplies current (assumed current) then it's positive:

    \[+V_a = \dots\]

  \item[-] The right hand side of the equation:

    The right hand side contains the current (assumed) multiplied by all elements' resistance it passes through:

    \[\ldots = I_a(R_{ab} + R_{ac}) \ldots\]

    If another current from a different mesh passes through some elements from the mesh started with:

    \[\ldots - I_b(R_{ab})\]

\end{itemize}

Therefore the full mesh equation:

\[+V_a = I_a(R_{ab} + R_{ac}) - I_b(R_{ab})\]

% TODO add examples

\section{Nodal analysis}

To find the voltages in a circuit using nodl analysis:

\begin{enumerate}

  \item Find the number of nodes in the circuit.
  \item Find the node connecting the most elements. (ground $\to V_{ref} = 0$)
  \item Apply KCL (Node equations) at each node to find the voltages.

\end{enumerate}

\begin{figure}[H]
  \begin{center}
    \begin{circuitikz}[scale = 0.7] \draw

      (0, 0) to[american current source, -o, l=$I_{s1}$] (0, 4)
      (0, 4) to[short, -o] (1, 4) to[R, l=$R_{12}$] (5, 4) to[short, o-] (6, 4)
      (6, 4) to[american current source, o-, l=$I_{s2}$] (6, 0)
      (6, 0) to[short, -o] (5, 0) to[short, -o] (3, 0) to[short, -o] (1, 0) to (0, 0)
      (5, 0) to[R, l=$R_{2}$] (5, 4) (1, 0) to[R, l_=$R_{1}$] (1, 4)
      (3, 0) node[ground] {}
      ;
    \end{circuitikz}
    \caption{Nodal analysis example}
  \end{center}
\end{figure}

\subsection{Node equations}

\begin{center} (Using the previous figure "Figure 1") \end{center}

\begin{itemize}

  \item[-] The left hand side of the equation:
    The left hand side contains the current (from a current source) at the node.

    If the current is entering the node then it's positive:

    \[+I_{s1} = \ldots\]

  \item[-] The right hand side of the equation:
    The right hand side contains the value of the current passing
    through each branch between two nodes in terms of the voltage and the resistance.

    \[\ldots = \frac{V_1 - 0}{R_1} + \frac{V_1 - V_2}{R_{12}}\]

\end{itemize}

Therefore the full node equation:

\[+I_{s1} = \frac{V_1 - 0}{R_1} + \frac{V_1 - V_2}{R_{12}}\]

\newpage

\subsection{Examples}

\begin{enumerate}

  \item Using mesh analysis, find the value and type
    of each power source given the currents a, b, c, d and e.

    \begin{center}

      \begin{circuitikz} \draw

        (0, 0) to[american current source, o-o, l=19A, i=$I_4$] (0, 2)
        (0, 2) to[short] (2, 2) to[short] (2, 4)
        (6, 4) to[american voltage source, o-o, i=$I_1$, l=$4I_d$] (2, 4)
        (6, 4) to[short, i=$I_e$] (6, 2)
        (6, 2) to[resistor, o-o, l=$10\Omega$, i_=$I_c$] (4, 2)
        (4, 2) to[resistor, o-o, l=$5\Omega$, i_=$I_b$] (2, 2)
        (2, 0) to[resistor, o-o, l=$40\Omega$, i=$I_3$] (2, 2)
        (0, 0) to[short] (2, 0) to[short, i_=$I_a$] (4, 0)
        (6, 0)to[short, i=$I_d$] (4, 0)
        (4, 0) to[american current source, o-o, l=$2I_b$] (4, 2)
        (6, 2) to[short, i=$I_2$] (6, 1.5)
        (6, 0) to[american voltage source, o-o, l_=240V] (6, 2)

        ;
      \end{circuitikz}

    \end{center}

    By assuming the currents $I_{1 \to 4}$ and their directions.
    \[I_4 = 19A. \qquad I_a = I_4 - I_3 = 19 - I_3. \qquad I_b = I_1 - I_3.\]
    \[I_c = I_1 - I_2. \qquad I_d = I_2. \qquad I_e = I_1\]

    Mesh equation at $I_1$ (1):
    \[4I_d = I_1 \times (10 + 5) - I_3 \times (5) - I_2 \times (10)\]
    \[\to 4I_2 = I_1 \times (10 + 5) - I_3 \times (5) - I_2 \times (10)\]

    Between $I_2$ and $I_3$ (2): (Super mesh)
    \[2I_b = 2I_1 - 2I_3 = I_2 - I_3\]
    \[\therefore 2I_1 - I_2 - I_3 = 0\]

    Super mesh equation (3):
    \[240 = I_2 \times (10) + I_3 \times (5 + 40) - I_4 \times (40) - I_1 \times (15)\]
    \[1000 = I_2 \times (10) + I_3 \times (5 + 40) - I_1 \times (15)\]

    By solving the three equations:

    \begin{center}

      \begin{tabular}{l c l}

        $15I_1 - 14I_2 - 5I_3 = 0$ & & $I_1 = 18A$ \\
        $2I_1 - I_2 - I_3 = 0$ & $\to$ & $I_2 = 10A$ \\
        $15I_1 - 10I_2 - 45I_3 = -1000$ & & $I_3 = 26A$

      \end{tabular}

    \end{center}

    \newpage

  \item Find the voltages across the following circuit.

    \begin{center}
      \begin{circuitikz} \draw

        (0, 2) to[american voltage source, o-o, l=100V] (0, 0)
        (0, 2) to[resistor, o-o, l=$1\Omega$] (0, 4)
        (0, 4) to[short, o-o] (0, 6) node[anchor=east] {$V_2$ :b}
        (0, 6) to[american current source, o-o, l=15A] (4, 6) node[anchor=west] {a: $V_1$}
        (4, 6) to[short, o-o] (4, 4)
        (4, 4) to[resistor, o-o, l=$4\Omega$] (0, 4)
        (4, 4) to[resistor, o-o, l=$8\Omega$] (6, 4)
        (4, 4) to[resistor, o-o, l=$4\Omega$] (4, 0)
        (6, 0) to[american voltage source, o-o, l=50V] (6, 4)
        (0, 0) to[short, o-o] (4, 0) node[ground] {} to[short] (6, 0)
        ;
      \end{circuitikz}
    \end{center}

    Nodal equation at node $a\ (V_1)$:
    \[15 = \frac{(V_1 + 50)}{8} + \frac{(V_1 - 0)}{4} + \frac{(V_1 - V_2)}{4}\]

    Nodal equation at node $b\ (V_2)$:
    \[-15 = \frac{(V_2 - V_1)}{4} + \frac{(V_1 - 100)}{1}\]

    By solving the two equations:
    \begin{center}

      \begin{tabular}{l c l}

        $120 = 5V_1 - 2V_2 + 50$ & $\to$ & $V_1 = -16.667$\\
        $60 = -5V_1 - V_2 - 100$ & & $V_2 = -76.667$

      \end{tabular}

    \end{center}

\end{enumerate}

\chapter{Technical theorems}
  \section{Superposition}
  \section{Source transfer}
  \section{Thevenin's theory}
  \section{Norton's theory}
\chapter{Energy storing elements}
\chapter{Diodes}
  \section{Diode operations}
  \section{Diode analysis}

\end{document}
