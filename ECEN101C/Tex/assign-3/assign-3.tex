\documentclass[a4paper]{report}
\usepackage{circuitikz}
\usepackage{amssymb}
\usepackage{amsmath}
\usepackage{multicol}
\ctikzset{bipoles/length=0.98cm}

\title{Assignment 3}
\author{Yahia Hany Gaber : 231000412}
\date{}

\begin{document}

\maketitle

\section{Problem 1}

\noindent a) Use the node-voltage method to find $i_1$, $i_2$ and $i_3$.

\noindent b) Find if the power dissipated in the circuit equals the power developed.

\begin{center}
	 

\end{center}

\begin{multicols}{2}
	 
a)

\noindent At $v_1$:
\[-10 \times 10^{-3} = \frac{v_1 - (-30)}{5000} + \frac{v_1 - 0}{500} + \frac{v_1 - 80}{1000}\]
\[\rightarrow v_1 = 20V\]
\[\therefore i_1 = \frac{v_1 - (-30)}{5000} = \frac{20 - (-30)}{5000} = 0.01A\]
\[\therefore i_2 = \frac{v_1 - 0}{500} = \frac{20}{500} = 0.04A\]

\[v_2 = v_3 = 80\]

\[i_4 = \frac{v_1 - 80}{1000} = \frac{20 - 80}{1000} = -0.06A\]
\[i_5 = \frac{80 - 0}{4000} = 0.02A\]

\noindent KCL at $v_3$:
\[\therefore 0.01 + i_4 = i_5 + i_3 \rightarrow i_3 = 0.01 + (-0.06) - 0.02\]
\[\rightarrow i_3 = -0.07A\]

\columnbreak

\begin{circuitikz} \draw

  (0, 0) to[american voltage source, l=30V] (0, 2)
  (0, 2) to[R=5k$\Omega$, -o] (2, 2)
  (2, 2) to[R=500$\Omega$, -o] (2, 0)
  (2, 2) to[american current source, o-o, l=10mA] (4, 2)
  (4, 2) to[R=4k$\Omega$, -o] (4, 0)
  (2, 2) to[short] (2, 4)
  (2, 4) to[R=1k$\Omega$, i=$i_4$] (6, 4)
  (6, 4) to[short, -o] (6, 2)
  (6, 2) to[short] (4, 2)
  (6, 2) to[american voltage source, l=80V] (6, 0)
  (6, 0) to[short] (0, 0)
  (1.8, 2) node[anchor=north, label={$v_1$}] {}
  (4, 2) node[anchor=north, label={$v_2$}] {}
  (6, 2) node[label={right:$v_3$}] {}
  (3, 0) node[ground] {}
  (0, 2) to[short, i=$i_1$] (0.5, 2)
  (2, 0.5) to[short, i=$i_2$] (2, 0.25)
  (6, 0.5) to[short, i=$i_3$] (6, 0.25)
  (4, 0.5) to[short, i=$i_5$] (4, 0.25)
  ;

\end{circuitikz}

\end{multicols}

b)

\noindent Power absorbed throughout the circuit:

  \[P_{5k\Omega} = I^2 \times R = (0.01)^2 \times 5000 = 0.5W\]
  \[P_{500\Omega} = (0.04)^2 \times 500 = 0.8W\]
  \[P_{4k\Omega} = (0.02)^2 \times 4000 = 1.6W\]
  \[P_{1k\Omega} = (0.06)^2 \times 1000 = 3.6W\]
  \[\therefore P_{abs} = 0.5 + 0.8 + 1.6 + 3.6 = 6.5W\]

\noindent Power delivered throught the circuit:
 \[P_{30V} = V \times I = 30 \times 0.01 = 0.3W\]
 \[P_{10mA} = (v_2 - v_1) \times I = 60 \times 0.01 = 0.6\]
 \[P_{80V} = 80 \times 0.07 = 5.6\]
 \[\therefore P_{del} = 0.3 + 0.6 + 5.6 = 6.5W\]

\[\therefore P_{abs} = P_{del} = 6.5W\]

\newpage

\section{Problem 2}

Apply mesh analysis to the circuit and find $I_o$

\begin{center}
	 
\begin{circuitikz} \draw

  (0, 0) to[R=$1\Omega$] (0, 3)
  (0, 6) to[american voltage source, l_=60V] (0, 3)
  (0, 6) to[R=$4\Omega$] (3, 6)
  (3, 6) to[R=$3\Omega$] (6, 6)
  (6, 6) to[R=$1\Omega$] (6, 3)
  (6, 3) to[american voltage source, l=22.5V] (6, 0)
  (6, 0) to[R=$4\Omega$] (3, 0)
  (0, 0) to[american current source, l_=5A] (3, 0)
  (3, 3) to[R=$1\Omega$, i=$I_o$] (3, 0)
  (3, 3) to[american current source, l_=10A] (3, 6)
  (3, 3) to[R=$2\Omega$] (6, 3)
  (3, 3) to[R=$2\Omega$] (0, 3);
  \draw[<-,shift={(1.5,1.5)}] (120:.7cm) arc (120:-90:.7cm) node at(0,0){$i_1$};
  \draw[<-,shift={(1.5,4.5)}] (120:.7cm) arc (120:-90:.7cm) node at(0,0){$i_2$};
  \draw[<-,shift={(4.5,4.5)}] (120:.7cm) arc (120:-90:.7cm) node at(0,0){$i_3$};
  \draw[<-,shift={(4.5,1.5)}] (120:.7cm) arc (120:-90:.7cm) node at(0,0){$i_4$};

\end{circuitikz}

\[I_o = i_1 - i_4\]

At $i_1$:
\[0 = i_1(1 + 2 + 1) - i_2(2) - i_4(1) \rightarrow (1)\]
Between $i_2$ and $i_3$ (Supermesh):
\[i_2 - i_3 = 10A \rightarrow (2)\]
After merging mesh 2 and 3:
\[-60 = i_2(2 + 4) + i_3(2 + 1 + 3) - i_1(2) - i_4(2) \rightarrow (3)\]
At $i_4$:
\[22.5 = i_4(4 + 2 + 1) - i_3(2) - i_1(1) \rightarrow (4)\]

By solving equations 1, 2, 3 and 4:
\[i_1 = \]

\end{center}

\end{document}
