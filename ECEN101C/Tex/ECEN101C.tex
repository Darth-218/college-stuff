\documentclass[a4paper]{book}
\usepackage{amssymb}
\author{Yahia Gaber}
\title{ECEN101C}
\begin{document}

\maketitle
\tableofcontents

\section{Introduction to electric circuits}

\subsection{Volt classification}

Electric volt can be classified into three categories: low, Medium and high.

\begin{center}

  \begin{tabular}{| c | c | c |}

    \hline
    Low voltage & Medium voltage & High voltage \\
    \hline
    :1000V & 1kV:50kV & 50kV:500kv \\
    \hline

  \end{tabular}

\end{center}

\subsection{Electric circuit components}

For an electric circuit to be one some elements need to be present.

\begin{enumerate}

  \item Power supplies
    A power supply is needed to create a potential difference to provide the circuit with a current.

  \item A load
    "loads" are energies converted in the circuit by elements that consume the power generated.

  \item Wires
    Wires are used to connect all the elements of the circuit together (Most commonly made of copper or aluminum).

  \item Switches
    Switches are used to "open" or "close" the circuit.

\end{enumerate}

\subsection{Basic circuit quantities}

\begin{enumerate}

  \item The electric charge (Coloumb)

  \item The electric current (Ampere)\\
    The current is the rate of change of the quantity of charge.

    \[I = \frac{dq}{dt}\]
    \[\therefore q = \int i(t).dt\]

    Example:

    \[q = 12e^{-12t} \to I = (-12)12e^{-12t}\]

  \item The potential difference (Volt $\to$ J/C "Joule per Coloumb")\\
    The potential difference is the energy affecting the charge moving it a certain distance.

    \[V = \frac{dw}{dq}\]

  \item The electric energy (Joule)

  \item The electric power (Watt $\to$ J/s "Joule per second")
    The power is the energy consumed/delivered in a certain period of time.

    \[P = \frac{dW}{dt} = I \times V = I^{2} \times R = \frac{V^2}{R}\]

\end{enumerate}

\subsection{Circuit types}

\begin{enumerate}

    % TODO Insert example

  \item DC circuits\\
    Direct current circuits are circuits with constant voltage and current.

  \item AC circuits\\
    Alternating current circuits are circuits with alternating voltage and current.

\end{enumerate}

\subsection{Basic circuit elements}

\begin{itemize}

    % TODO Insert example

  \item Passive elements \\
    Elements that absorb the power generated by the active elements.\\
    Some of the common passive elements seen in circuits:

    \begin{center}
      \begin{tabular}{| c | c | c |}

        \hline
        Element & Unit & Found in \\ \hline \hline
        Resistors & Ohm ($\Omega$) & AC and DC \\ \hline
        Electrical coils & Henry (L) & AC only \\ \hline
        Capacitors & Farad (C) & AC only\\
        \hline

      \end{tabular}
    \end{center}

  \item Active elements \\ Elements that generate power for the circuit.

    \begin{itemize}

      \item Current sources

        \begin{enumerate}

          \item Independant Current sources

            Independant sources provide constant current intensity.

          \item Dependant current sources

            Dependant sources have variable current intensities and are either:

            \begin{itemize}

              \item Voltage controlled ($V_x$)

              \item Current controlled ($I_y$)

            \end{itemize}

        \end{enumerate}

    \end{itemize}

\end{itemize}

\section{Basic circuit laws}

\subsection{Ohm's law}

\[V = I \times R\]

\subsubsection{Power types and the conventional sign rule}

\begin{itemize}

  \item[-] Power absorbed

    Power is absorbed when the current's direction is into the positive terminal

    % TODO Insert example

  \item[-] Power supplied

    Power is supplied when the current's direction is into the negative terminal

    % TODO Insert example

\end{itemize}

For any balanced (ideal) circuit, the sum of the power consumed equals the sum of power absorbed.

\[\Sigma P_{abs} = \Sigma P_{con}\]

\noindent Note: The conventional sign rule is only applied when both the current 
and the voltage are positive and allows the switching of either the direction of 
a current of the terminals of a voltage source in case the magnitude is a negative 
value.

\subsubsection{Example}

% TODO Add example from lec

\subsection{Kirchhoff's laws}

\subsubsection{Kirchhoff's voltage law}

In any loop in a circuit, the sum of voltages across the loop equals zero

\[\sum V_{loop} = 0\]

\subsubsection{Series connection}

Circuit elements are in series only if the same current intensity passes through them as the voltage is divided between the elements (not equally).\\
Therefore the equivalent resistance of a number of resistors in series is:

\[R_{eq} = \sum_{n = 1}^r R_{n}\]

\subsubsection{Voltage division}

As the voltage is divided between the elements in series in non-uniform quantities, the voltage of each element can be found as the voltage 
is directly proportional with the value of the resistance of each element. Therefore:

\[V_a = V_t \times \frac{R_a}{R_{eq}}\]

\subsubsection{Kirchhoff's current law}

\begin{itemize}

  % TODO put the list items in a box

  \item[*] Junctions: points of connection that connect only two circuit elements.

  \item[*] Nodes: points of connection that connect more than two circuit elements.

\end{itemize}

\noindent At any node, the sum of currents with a direction into the node equals to the sum of
currents with a direction outside the node.

\[\sum I_{in} = \sum I_{out}\]

\subsubsection{Parallel connection}

Circuit elements are in parallel only if they share the same starting and ending node as the current is 
divided between each element while the voltage remains the same.\\
Therefore the equivalent resistance of a number of resistors in series is:

\[\frac{1}{R_{eq}} = \sum_{n=1}^r \frac{1}{R_n}\]

\subsubsection{Current division}

As the current is devided between the elements in parallel non-uniform quantities, the current through
each element can be found as the current intensity is inversly proportional with the value of the resistance of each element. Therefore:

\[I_a = I_{t} \times \frac{R_b}{R_a + R_b}\]

\subsubsection{Conductance}

Conductance ($G$) is the receprocal quantity to the electrical resistance and is measured in siemens ($S$)

\[G = \frac{I}{V} = \frac{1}{R}\]

\section{Techincal methods for solving electrical circuits}

  \subsection{Nodal analysis}

  \subsection{Mesh analysis}

\section{Technical theorems}
  \subsection{Superposition}
  \subsection{Source transfer}
  \subsection{Thevenin's theory}
  \subsection{Norton's theory}
\section{Energy storing elements}
\section{Diodes}
  \subsection{Diode operations}
  \subsection{Diode analysis}

\end{document}
