% Created 2024-02-21 Wed 18:23
% Intended LaTeX compiler: pdflatex
\documentclass[11pt]{article}
\usepackage[utf8]{inputenc}
\usepackage[T1]{fontenc}
\usepackage{graphicx}
\usepackage{longtable}
\usepackage{wrapfig}
\usepackage{rotating}
\usepackage[normalem]{ulem}
\usepackage{amsmath}
\usepackage{amssymb}
\usepackage{capt-of}
\usepackage{hyperref}
\author{Yahia Gaber}
\date{\today}
\title{}
\hypersetup{
 pdfauthor={Yahia Gaber},
 pdftitle={},
 pdfkeywords={},
 pdfsubject={},
 pdfcreator={Emacs 29.1 (Org mode 9.7)}, 
 pdflang={English}}
\begin{document}

\tableofcontents

date: 20-2-23
\section{Lecture \#2}
\label{sec:org65ce475}

\subsection{Basic circuit laws (cont.)}
\label{sec:orgaf6eb8c}

Example on Ohm's law:

A voltage supply of 100V supplies 500W to 4 elements, 10, 12 and 2 in ohm.
Find the power absorbed by the 4th element.

$$I_t = \frac{P}{V} = 5A$$
$$P_1 = 5^2 \times 10 = 250W$$
$$P_2 = 3^2 \times 12 = 108W$$
$$P_3 = 2^2 \times 2 = 8W$$
$$\therefore P_4 = P - (P_1 + P_2 + P_3) = 500 - 366 = 134W$$
\subsubsection{Kirchhoff's Law}
\label{sec:org9e7a8ab}

\begin{enumerate}
\item 1. KVL (Kirchhoff's voltage law)
\label{sec:org2900cc3}

In any loop inside a circuit, The sum of voltage across the loop equals zero.

$$\Sigma V_{loop} = 0$$

\begin{itemize}
\item Series connection

Elements are in series connection only if the same current passes through all of them. Therefore the equivelant resistance of n resistors in series equals:

$$R_{eq} = \Sigma_{n = 1} R_n$$

\item Voltage division

$$V = V_t \times \frac{R_1}{R_t}$$

\begin{itemize}
\item When in a series connection, The value of the voltage is directly proportional with the value of the resistance
\end{itemize}
\end{itemize}
\item 2. KCL (Kirchhoff's current law)
\label{sec:org23c984e}

\begin{itemize}
\item Junction

A junction is a connection point in the circuit between only two elements and the current passing through is constant.

\item Node:

A node is a connection point in the circuit that connects at least three branches and is where the current is branched.
\end{itemize}

KCL states that the sum of currents going into a node equals the sum of currents going out of the same node

$$\Sigma I_{in} = \Sigma I_{out}$$

\begin{itemize}
\item Parallel connection

Parallel connection is where elements share the same start and end point and therefore share the same voltage

$$\frac{1}{R_{eq}} = \Sigma_{n = 1} \frac{1}{R_n}$$

\begin{itemize}
\item In the case of only two resistors:

$$R_{eq} = \frac{R_1 \times R_2}{R_1 + R_2}$$
\end{itemize}

\item Current division

$$\frac{I_1}{I_2} = \frac{R_2}{R_1}$$

$$I_1 = I_t \times \frac{R_2}{R_1 + R_2}$$

\begin{itemize}
\item When in parallel connection, the value of the current is inversly proportional with the value of the resistance.
\end{itemize}
\end{itemize}
\end{enumerate}
\subsubsection{Conductance (\(G\))}
\label{sec:org3b24d52}

$$G = \frac{1}{R} = \omega = S$$
\end{document}
