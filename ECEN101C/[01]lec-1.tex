% Created 2024-02-21 Wed 18:24
% Intended LaTeX compiler: pdflatex
\documentclass[11pt]{article}
\usepackage[utf8]{inputenc}
\usepackage[T1]{fontenc}
\usepackage{graphicx}
\usepackage{longtable}
\usepackage{wrapfig}
\usepackage{rotating}
\usepackage[normalem]{ulem}
\usepackage{amsmath}
\usepackage{amssymb}
\usepackage{capt-of}
\usepackage{hyperref}
\usepackage{amsmath}
\author{Yahia Gaber}
\date{\today}
\title{}
\hypersetup{
 pdfauthor={Yahia Gaber},
 pdftitle={},
 pdfkeywords={},
 pdfsubject={},
 pdfcreator={Emacs 29.1 (Org mode 9.7)}, 
 pdflang={English}}
\begin{document}

\tableofcontents

date: 13-2-23
\section{Lecture 1}
\label{sec:orge89a20d}

\subsection{Volt calssification}
\label{sec:org99725ad}

\begin{itemize}
\item up to 1000V            >> Low voltage

\item between 1kV and 50kV   >> Mideum voltage

\item between 50kV and 500kV >> High voltage
\end{itemize}
\subsection{Electric circuits}
\label{sec:orgc0596e9}

For the circuits to be a circuit it needs:

\begin{itemize}
\item A supply

\item A load

\begin{itemize}
\item Light

\item Heat

\item Mechanical energy aka a motor
\end{itemize}

\item A wire

\item A switch
\end{itemize}
\subsection{Basic circuit quantities}
\label{sec:org50e57b8}

Most common types of wire are copper (Cu) and aluminum (Al).

\begin{enumerate}
\item The charge  (q > coulumb)

\item The current (I > ampere = C/S)

The rate of change of the quantity of charges.

\(I = \tfrac{dq}{dt}\)   > differentiation of the charge

\(q = \int i(t).dt\) > integration of the current

Example:

\(q = 12e-12t \rightarrow I = 12(-12)e-12t mA\)

\item The potential difference (V > volt = J/C)

The energy affecting the charge to move it a certain distance.

\(V = \frac{dW}{dq}\)

\item The electrical energy (W > joule)

\item The electrical power (P > watt = J/S)

\(p = \frac{dW}{dt} = I \times V = V \times I = I^{2} \times R = \frac{V^{2}}{R}\)
\end{enumerate}
\subsection{Basic circuit analysis}
\label{sec:orgb8ed15e}

\begin{enumerate}
\item DC circuits

Constant voltage and current.

\item AC circuits

Alternating voltage and current.
\end{enumerate}
\subsection{Basic circuit elements}
\label{sec:org6d0a1cf}

\begin{enumerate}
\item Passive elements

Elements that absorb the electrical energy.

\begin{itemize}
\item The most common passive elements:

\begin{itemize}
\item The resistor        (R > ohm)    (DC \& AC)
\item The electrical coil (L > henery) (AC only)
\item The capacitor       (C > farad)  (AC only)
\end{itemize}
\end{itemize}

\item Active elements

Elements that generate the electrical energy.

\begin{enumerate}
\item Current source

Identified from the direction.

\begin{itemize}
\item Independant

Constant current intensity.

\item Dependant (Voltage controlled -Vx- || Current controlled -Iy-)

Variable current intensity.
\end{itemize}

\item Voltage source

Identified from the polarity.

\begin{itemize}
\item Independant

Constant voltage

\item Dependant (Voltage controlled -Vy- || Current controlled -Io-) 

Variable voltage
\end{itemize}
\end{enumerate}
\end{enumerate}
\subsection{Basic circuit laws}
\label{sec:orgec8564e}

\begin{enumerate}
\item Ohm's law

\(V = I \times R\)

\begin{itemize}
\item Power types (Conventional sign rule)

Only applied when both the current and voltage are positive.

\begin{enumerate}
\item Power absorbed

Current enters the element from the positive side.

\item Power supplied/delivered

Current leaves the element from the positive side.
\end{enumerate}
\end{itemize}

For any balanced circuit (with an ideal wire) the sum of the power consumed equals the sum of the supplied power

\item Kirchoff's law

[NEXT LECTURE]
\end{enumerate}
\end{document}
