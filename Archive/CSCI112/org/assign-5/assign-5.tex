\documentclass{report}
\usepackage{listings}

\begin{document}

\chapter{Assignment \#5}

Student Name: Yahia Hany Gaber | Studnet ID: 231000412

\section{Question 1}

\begin{lstlisting}{C}
#include <stdio.h>

int max(int *arr, int start, int end, int maxind);

int main() {

  int limit;

  printf("Enter the number of array elements: ");
  scanf("%d", &limit);

  int arr[limit];
  for (int x = 0; x < limit; x++) {

    int a;
    printf("Enter element %d: ", x + 1);
    scanf("%d", &a);
    arr[x] = a;
  }

  printf("The maximum element is: %d\n", max(arr, 0, limit, 0));
}

int max(int *arr, int start, int end, int maxind) {

  int _max = arr[maxind];

  if (start < end) {

    if (arr[start] > _max) {

      _max = arr[start];
      max(arr, start + 1, end, start);
    }
    else {

      max(arr, start + 1, end, maxind);
    }
  }
  else {return _max;}
}
\end{lstlisting}

\section{Question 2}

\begin{lstlisting}{C}
    #include <stdio.h>

    int sum(int *arr, int start, int end, int startsum);

    int main() {

      int limit;

      printf("Enter the number of array elements: ");
      scanf("%d", &limit);

      int arr[limit];
      for (int x = 0; x < limit; x++) {

        int a;
        printf("Enter element %d: ", x + 1);
        scanf("%d", &a);
        arr[x] = a;
      }

      printf("The sum of the array's elements is: %d\n",
        sum(arr, 0, limit, 0));
    }

    int sum(int *arr, int start, int end, int startsum) {

      if (start < end) {

        startsum += arr[start];
        sum(arr, start + 1, end, startsum);
      }
      else {return startsum;}
    }
\end{lstlisting}

\section{Question 3}

\begin{lstlisting}{C}
   #include <stdio.h>

   #define new printf("\n")

   void reverse(int *arr, int end);

   int main() {

     int limit;

     printf("Enter the number of array elements: ");
     scanf("%d", &limit);

     int arr[limit];
     for (int x = 0; x < limit; x++) {

       int a;
       printf("Enter element %d: ", x + 1);
       scanf("%d", &a);
       arr[x] = a;
     }

     reverse(arr, limit - 1);
   }

   void reverse(int *arr, int end) {

     if (end >= 0) {

       printf("%d ", arr[end]);
       reverse(arr, end - 1);
     }
     else {new; return;}
   }
\end{lstlisting}

\section{Question 4}

\begin{lstlisting}{C}
   #include <stdio.h>

   int secondlargest(int *arr, int end);

   int main() {

     int limit;

     printf("Enter the number of array elements: ");
     scanf("%d", &limit);

     int arr[limit];
     for (int x = 0; x < limit; x++) {

       int a;
       printf("Enter element %d: ", x + 1);
       scanf("%d", &a);
       arr[x] = a;
     }
     printf("The second largest element in the array is: %d",
           secondlargest(arr, limit));
   }

   int secondlargest(int *arr, int end) {

     int temp;

     for (int x = 0; x < end; x++) {

       for (int y = 0; y < end; y++) {

         if (arr[x] < arr[y]) {

           temp = arr[x];
           arr[x] = arr[y];
           arr[y] = temp;
         }
       }
     }

     return arr[end - 2];
   }
\end{lstlisting}

\section{Question 5}

\begin{lstlisting}{C}
#include <stdio.h>

int main() {

  char arr1[3] = {1, 2, 3};
  char arr2[3] = {10, 12, 14};
  int arr3[6];
  for (int i = 0; i < 3; i++) {
    arr3[i] = arr1[i];
  }
  for (int i = 0; i < 6; i++) {
    arr3[i] = arr1[i];
  }
}
\end{lstlisting}

\end{document}
